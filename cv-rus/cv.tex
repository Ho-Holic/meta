\documentclass[a4paper]{curricula-vitae}
\usepackage{fontspec}
\usepackage[russian]{babel}
 
 % Конец страницы должен содержать подвал с моим именем, это позволяет легче 
 % найти резюме еслит оно лежит в стопке других
\footersection{\today}{Игорь Сысоев~~~·~~~Резюме}{} % \thepage если будет несколько страниц

\begin{document}

%----------------------------------------------------------------------------------------
%  First Page
%----------------------------------------------------------------------------------------

%----------------------------------------------------------------------------------------
% TITLE SECTION
%----------------------------------------------------------------------------------------

\namesection{Сысоев Игорь}

%----------------------------------------------------------------------------------------
% LEFT COLUMN
%----------------------------------------------------------------------------------------
\begin{minipage}[t]{0.33\textwidth} % занимает 33% от общего объема

\section{Контакты} 

\subsection{Почта}
igor.sysoev.main@gmail.com

\insertspace

\subsection{Телефон}
+7(915)388-51-91

%\subsection{Профили}
%google-plus
%git-hub
%stack-overflow

\insertspace

\section{MOOC}

\subsection{Coursera}
Algorithms: Design and Analysis \\
Part 1 \textbullet{} Algorithms, Part 1 \& 2 \\ 
Automata \textbullet{} Functional \\
Programming Principles in Scala

\insertspace

\subsection{EdX}
Paradigms of Computer \\
Programming

\insertspace

\section{Навыки}

\subsection{Программирование}
C++14 \textbullet{} C++/Templates \\
STL \textbullet{} Qt \textbullet{} PostgreSQL


\insertspace

\subsection{Веб}
HTML \textbullet{} CSS \textbullet{} JavaScript

\insertspace

\subsection{Проект}
Unix/Scripting \textbullet{} Git \textbullet{} Svn \\
Redmine \textbullet{} Onepoint Projects \\
Wiki/Markdown \textbullet{} \LaTeX

\insertspace

\subsection{Дополнительно}
Photoshop \textbullet{} Cockos Reaper \\
Maya

\insertspace

\subsection{Языки}
Русский \textbullet{} Английский

\insertspace

\subsection{О себе}
Готов к релокации \\
T-Shaped skills

\insertspace

\end{minipage} % конец левой колонки
\hfill
% Не оставляйте пустую строку между двумя minipage - полетит вёрстка!
%----------------------------------------------------------------------------------------
% RIGHT COLUMN
%----------------------------------------------------------------------------------------
\begin{minipage}[t]{0.66\textwidth} % занимает 66% от общего объема

\section{Опыт работы} 

\mainsubsection{Мобильные информационные системы}
\addinfo{Руководитель группы}

\workperiod{May 2013 – Настоящий момент}{Москва}
\vspace{\topsep} % !!! хак убрирает глюк с наезжанием пунктов на заголовок !!!
\begin{tightitemize}
\item Создание архитектуры и инфраструктуры
\item Создание системы сборки
\item Разработка стандарта кодирования
\item Clang Static Analyzer, cppcheck
\item Генерация отчётов и создание общирного количества тестов в Qt Test Framework
\item Настройка/генерация doxygen
\item Написание и поддержка внутресетевых веб ресурсов, в основном для отображеня
информации о здоровье проекта (отчёты анализаторов и тестов)
\item Написание абстрактного интерфейся оборачивающего две доступные геодизические библиотеки
\end{tightitemize}

\insertspace

\mainsubsection{Мобильные информационные системы}
\addinfo{Разработчик}

\workperiod{May 2013 – Aug 2013}{Москва}
Большую часть времени занимался созданием UI и
решением расчётных задач окологеодезической тематики.
\begin{tightitemize}
\item работа с геодезическими фреймворками
\item Обширное использование шаблонов
\item Создание DSL для решения расчётных задач
\item Нашел с помошью valgrind падение которое обваливало продукт при закрытии долгие годы
\item Получил 6 кью играя в Го с коллегами на обедах % здесь должна быть смешная штука
\end{tightitemize}

\insertspace

\section{Образование} 

\mainsubsection{Московский авиационный институт}
\addinfo{Магистр}

\workperiod{May 2013 – Aug 2013}{Москва}
Создание нейронной сети на Qt/C++, На выбор темы диплом повлиял курс ML Andry Ng, Octave,
Межпроцессное взаимодействие

\insertspace

\mainsubsection{Московский авиационный институт}
\addinfo{Бакалавр}

\workperiod{May 2013 – Aug 2013}{Москва}
Диплом писался совместно. Больше 20к строк кода (уточнить). Использование Bazar.
MsSql в качестве базы. Работа из кода с форматом Microsoft Excel.
Сетевая программа где пользователи имеют разные уровни доступа.
ISO 9000, ВНИИ ГАЗ, C\#

\insertspace

\end{minipage} % right column end

%----------------------------------------------------------------------------------------
% SECOND PAGE
%----------------------------------------------------------------------------------------

\end{document} % document end
