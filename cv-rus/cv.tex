\documentclass[a4paper]{curricula-vitae}
\usepackage{fontspec}
 
\begin{document} % document begin

%----------------------------------------------------------------------------------------
% FIRST PAGE
%----------------------------------------------------------------------------------------

%----------------------------------------------------------------------------------------
% TITLE SECTION
%----------------------------------------------------------------------------------------

\namesection{\{ Игорь Сысоев \}}

%----------------------------------------------------------------------------------------
% LEFT COLUMN
%----------------------------------------------------------------------------------------
\begin{minipage}[t]{0.33\textwidth} % takes 33%

\section{Контакты} 
\subsection{почта}
igor.sysoev.main@gmail.com
\subsection{телефон}
+7(915)388-51-91

\section{MOOC}
\subsection{Coursera}
Algorithms, Part I and II/Princeton
Algorithms: Design and Analysis, Part 1/Standford
Automata/Standford
Functional Programming Principles in Scala/EPFL

\subsection{EdX}
Paradigms of Computer Programming

\section{Навыки}

\subsection{Программирование}
C++14, Stl, Qt,
Postgresql, Sfml

\subsection{Веб}
HTML, CSS, JavaScript

\subsection{Языки}
Русский, Английский

\subsection{Проект}
Unix/Scripting, Git, Svn
Redmine, Onepoint Projects
Wiki, TeX

\subsection{Дополнительно}
Photoshop, 
Cockos Reaper, 
Maya

\end{minipage} % left column end
\hfill
% No blank lines between two minipages!
%----------------------------------------------------------------------------------------
% RIGHT COLUMN
%----------------------------------------------------------------------------------------
\begin{minipage}[t]{0.66\textwidth} % takes 66%

\section{Опыт работы} 

\mainsubsection{Мобильные информационные системы}
\addinfo{:: Руководитель группы}

\workperiod{May 2013 – Настоящий момент | Москва}
\begin{tightitemize}
\item Создание архитектуры и инфраструктуры
\item Создание системы сборки
\item Разработка стандарта кодирования
\item Clang Static Analyzer, cppcheck
\item QTestingFramework
\item Написание и поддержка внутресетевых веб ресурсов, в основном для отображеня
информации о здоровье проекта (отчёты анализаторов и тестов)
\item Работа с геодизическими системами
\end{tightitemize}

\mainsubsection{Мобильные информационные системы}
\addinfo{:: Разработчик}

\workperiod{May 2013 – Aug 2013 | Москва}
\begin{tightitemize}
\item Обширное использование шаблонов
\item Создание DSL для решения расчётных задач
\end{tightitemize}

\section{Образование} 

\mainsubsection{Московский авиационный институт}
\addinfo{:: Магистр}

\workperiod{May 2013 – Aug 2013 | Москва}
Создание нейронной сети на Qt/C++, На выбор темы диплом повлиял курс ML Andry Ng, Octave,
Межпроцессное взаимодействие

\mainsubsection{Московский авиационный институт}
\addinfo{:: Бакалавр}

\workperiod{May 2013 – Aug 2013 | Москва}
Диплом писался совместно. Больше 20к строк кода (уточнить). Использование Bazar.
MsSql в качестве базы. Работа из кода с форматом Microsoft Excel.
Сетевая программа где пользователи имеют разные уровни доступа.
ISO 9000, ВНИИ ГАЗ, C\#

\end{minipage} % right column end

%----------------------------------------------------------------------------------------
% SECOND PAGE
%----------------------------------------------------------------------------------------

\end{document} % document end
