\documentclass[a4paper]{curricula-vitae}
\usepackage{fontspec}
\usepackage[russian]{babel}
 
 % Конец страницы должен содержать подвал с моим именем, это позволяет легче 
 % найти резюме еслит оно лежит в стопке других
\footersection{\today}{Игорь Сысоев~~~·~~~Резюме}{
%\thepage %пока только одна страница
}

% В правом углу страницы можно сделать \thepage если будет несколько страниц 
% или QR code с сылкой на актуальное резюме, профиль гугл плюс или rick roll (=

\begin{document}

%----------------------------------------------------------------------------------------
%  Первая страница
%----------------------------------------------------------------------------------------

%----------------------------------------------------------------------------------------
% Заголовок
%----------------------------------------------------------------------------------------

% options for first section: \#, greek omega or lambda letter, n * log(n), infinity symbol

\namesection{}{Сысоев}{Игорь}
{Разработчик программного обеспечения}

%----------------------------------------------------------------------------------------
% Левая колонка
%----------------------------------------------------------------------------------------
\begin{minipage}[t]{0.33\textwidth}%

\section{Контакты} 

\subsection{Почта}
\href{mailto:igor.sysoev.main@gmail.com}{igor.sysoev.main@gmail.com}

\insertspace

\subsection{Телефон}
\href{tel:+79153885191}{+7 (915) 388 - 51 - 91}

\insertspace

\subsection{Профили}
\href{https://plus.google.com/+%D0%98%D0%B3%D0%BE%D1%80%D1%8C%D0%A1%D1%8B%D1%81%D0%BE%D0%B5%D0%B2}{Google Plus}
\textbullet{} \href{https://t.me/hoholic}{Telegram}
%\href{https://github.com/ho-holic}{GitHub}
%\href{http://catbox.io}{Персональный сайт}

\insertspace

\section{MOOC}

\subsection{Coursera}
Algorithms: Design and Analysis \\
Part 1 \textbullet{} Algorithms, Part 1 \& 2 \\ 
Automata \textbullet{} Functional \\
Programming Principles in Scala

\insertspace

\subsection{EdX}
Paradigms of Computer \\
Programming

\insertspace

\section{Навыки}

\subsection{Программирование}
C++17 \textbullet{} C++/Templates \\
STL \textbullet{} Qt \textbullet{} PostgreSQL \\
OpenGL \textbullet{} AndroidNDK


\insertspace

\subsection{Веб}
HTML \textbullet{} CSS \textbullet{} JavaScript

\insertspace

\subsection{Проект}
Unix/Scripting \textbullet{} XSL \textbullet{} Git \\
Redmine \textbullet{} Onepoint Projects \\
Wiki/Markdown \textbullet{} \LaTeX \\
GitLab

\insertspace

\subsection{Дополнительно}
Photoshop \textbullet{} Cockos Reaper \\
Maya

\insertspace

\subsection{Языки}
Русский \textbullet{} Английский

\insertspace

\subsection{О себе}
Готов к переезду \\
T-Shaped skills
% Game of Go (4k)

\insertspace
\insertspace

% нормальная ссылка на профиль
%\qrcodesection{https://plus.google.com/+%D0%98%D0%B3%D0%BE%D1%80%D1%8C%D0%A1%D1%8B%D1%81%D0%BE%D0%B5%D0%B2}

% немного троллинга =)
\qrcodesection{https://www.youtube.com/watch?v=dQw4w9WgXcQ&autoplay=1}

\end{minipage}%
\hfill
% Не оставляйте пустую строку между двумя minipage - полетит вёрстка!
%----------------------------------------------------------------------------------------
% Правая колонка
%----------------------------------------------------------------------------------------
\begin{minipage}[t]{0.66\textwidth}%

\section{Опыт работы} 

\workplace{Мобильные информационные системы}{Фев 2014 – Настоящий момент}{Руководитель группы}
Основными задачами было создание архитектуры и инфраструктуры приложения
\vspace{\topsep} % !!! хак убрирает глюк с наезжанием пунктов на заголовок !!!
\begin{tightitemize}
\item разработка стандарта кодирования
\item система автоматической сборки приложения
\item статическое тестирование с помощью Сlang Static Analyzer и Cppcheck
\item генерация отчётов с помошью XSL и создание обширного количества тестов в Qt Test Framework
\item генерация документации с помощью Doxygen
\item написание и поддержка внутресетевых веб ресурсов, в основном для отображеня
информации о здоровье проекта (отчёты анализаторов и тестов)
\item написание абстрактного интерфейса оборачивающего две доступные геодизические библиотеки
\end{tightitemize}

\insertspace

\workplace{Мобильные информационные системы}{Мар 2010 – Фев 2014}{Разработчик}
Большую часть времени занимался созданием Ui и
решением расчётных задач окологеодезической тематики
\begin{tightitemize}
\item работа с геодезическими фреймворками
\item обширное использование C++/Templates
\item создание DSL для решения расчётных задач
\item нашел с помошью valgrind падение которое обваливало продукт при закрытии долгие годы
%\item получил 6 кью играя в Го с коллегами на обедах % здесь должна быть смешная шутка
\end{tightitemize}

\insertspace

\section{Образование} 

\workplace{Московский авиационный институт}{Сен 2010 – Июн 2012}{Магистр}
Создание нейронной сети для решения задач прогнозирования расхода ресурсов. 
На выбор темы диплома повлиял курс по машинному обучению от Andrew Ng
\begin{tightitemize}
\item написана нейронная сеть в виде скриптов на языке Octave
\item написан клиент на C++/Qt осуществляющий межпроцессное взаимодействие с интерпретатором 
языка Octave для передачи данных и получения результатов работы сети
%\item реализована система рисования полученных данных через канву
\end{tightitemize}

\insertspace

\workplace{Московский авиационный институт}{Сен 2006 – Июн 2010}{Бакалавр}
На протяжении года совместно с одногрупником и руководителем для <<Газпром ВНИИГАЗ>> была написана
программа заменяющая ручную работу по контролю качества производимую в Excel
\begin{tightitemize}
\item больше 20к строк кода на C\#/.Net Framework
\item работа через API с документами Microsoft Excel (импортированы ранее существующие документы)
\item сетевое взаимодействие между пользователями которые имеют разные уровни доступа
\item использование Bazar как системы контроля версий и Microsoft SQL в качестве базы данных
\end{tightitemize}

\insertspace

\end{minipage}%

%----------------------------------------------------------------------------------------
% Вторая страница (в будущем)
%----------------------------------------------------------------------------------------

\end{document}
