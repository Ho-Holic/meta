\documentclass[a4paper]{curricula-vitae}
\usepackage{fontspec}
 
 % Конец страницы должен содержать Footer с моим именем, это позволяет легче 
 % найти резюме еслит оно лежит в стопке

\begin{document} % document begin

%----------------------------------------------------------------------------------------
% FIRST PAGE
%----------------------------------------------------------------------------------------

%----------------------------------------------------------------------------------------
% TITLE SECTION
%----------------------------------------------------------------------------------------

\namesection{\{ Игорь Сысоев \}}

%----------------------------------------------------------------------------------------
% LEFT COLUMN
%----------------------------------------------------------------------------------------
\begin{minipage}[t]{0.33\textwidth} % занимает 33% от общего объема

\section{Контакты} 

\subsection{Почта}
igor.sysoev.main@gmail.com

\insertspace

\subsection{Телефон}
+7(915)388-51-91

\insertspace

\section{MOOC}

\subsection{Coursera}
Algorithms, Part I and II/Princeton \\
Algorithms: Design and Analysis \\
Part 1/Standford \\
Automata/Standford \\
Functional Programming Principles \\
in Scala/EPFL

\insertspace

\subsection{EdX}
Paradigms of Computer \\
Programming

\insertspace

\section{Навыки}

\subsection{Программирование}
C++14 \textbullet{} C++/Templates \\
STL \textbullet{} Qt \\
PostgreSQL

\insertspace

\subsection{Веб}
HTML \textbullet{} CSS \textbullet{} JavaScript

\insertspace

\subsection{Проект}
Unix/Scripting \textbullet{} Git \textbullet{} Svn \\
Redmine \textbullet{} Onepoint Projects \\
Wiki/Markdown \textbullet{} \LaTeX

\insertspace

\subsection{Дополнительно}
Photoshop \textbullet{} Cockos Reaper \\
Maya

\insertspace

\subsection{Языки}
Русский \textbullet{} Английский

\insertspace

\end{minipage} % конец левой колонки
\hfill
% Не оставляйте пустую строку между двумя minipage - полетит вёрстка!
%----------------------------------------------------------------------------------------
% RIGHT COLUMN
%----------------------------------------------------------------------------------------
\begin{minipage}[t]{0.66\textwidth} % занимает 66% от общего объема

\section{Опыт работы} 

\mainsubsection{Мобильные информационные системы}
\addinfo{Руководитель группы}

\workperiod{May 2013 – Настоящий момент | Москва}
\vspace{\topsep} % хак убрирает глюк с наезжанием пунктов на заголовок
\begin{tightitemize}
\item Создание архитектуры и инфраструктуры
\item Создание системы сборки
\item Разработка стандарта кодирования
\item Clang Static Analyzer, cppcheck
\item Генерация отчётов и создание общирного количества тестов в Qt Test Framework
\item Настройка/генерация doxygen
\item Написание и поддержка внутресетевых веб ресурсов, в основном для отображеня
информации о здоровье проекта (отчёты анализаторов и тестов)
\item Работа с геодизическими системами
\end{tightitemize}

\insertspace

\mainsubsection{Мобильные информационные системы}
\addinfo{Разработчик}

\workperiod{May 2013 – Aug 2013 | Москва}
\begin{tightitemize}
\item Обширное использование шаблонов
\item Создание DSL для решения расчётных задач
\end{tightitemize}

\insertspace

\section{Образование} 

\mainsubsection{Московский авиационный институт}
\addinfo{Магистр}

\workperiod{May 2013 – Aug 2013 | Москва}
Создание нейронной сети на Qt/C++, На выбор темы диплом повлиял курс ML Andry Ng, Octave,
Межпроцессное взаимодействие

\insertspace

\mainsubsection{Московский авиационный институт}
\addinfo{Бакалавр}

\workperiod{May 2013 – Aug 2013 | Москва}
Диплом писался совместно. Больше 20к строк кода (уточнить). Использование Bazar.
MsSql в качестве базы. Работа из кода с форматом Microsoft Excel.
Сетевая программа где пользователи имеют разные уровни доступа.
ISO 9000, ВНИИ ГАЗ, C\#

\insertspace

\end{minipage} % right column end

%----------------------------------------------------------------------------------------
% SECOND PAGE
%----------------------------------------------------------------------------------------

\end{document} % document end
